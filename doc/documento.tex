\documentclass{itam}
\usepackage{mystyle}
\usepackage{Sweave}
\begin{document}
\Sconcordance{concordance:documento.tex:documento.Rnw:%
1 2 1 1 0 7 1}
\Sconcordance{concordance:documento.tex:./tex/intro.Rnw:ofs 11:%
1 27 1}
\Sconcordance{concordance:documento.tex:documento.Rnw:ofs 39:%
12 1 1}
\Sconcordance{concordance:documento.tex:./tex/requisitos.Rnw:ofs 41:%
1 4 1}
\Sconcordance{concordance:documento.tex:documento.Rnw:ofs 46:%
15 1 1}
\Sconcordance{concordance:documento.tex:./tex/instalacion.Rnw:ofs 48:%
1}
\Sconcordance{concordance:documento.tex:documento.Rnw:ofs 49:%
18 1 1}
\Sconcordance{concordance:documento.tex:./tex/instrucciones.Rnw:ofs 51:%
1}
\Sconcordance{concordance:documento.tex:documento.Rnw:ofs 52:%
21 1 1}
\Sconcordance{concordance:documento.tex:./tex/descripcion.Rnw:ofs 54:%
1 3 1}
\Sconcordance{concordance:documento.tex:documento.Rnw:ofs 58:%
24 1 1}
\Sconcordance{concordance:documento.tex:./tex/referencias.Rnw:ofs 60:%
1 12 1}
\Sconcordance{concordance:documento.tex:documento.Rnw:ofs 73:%
27 1 1}


\title{Interfaz remota para la administración de usuarios de un Directorio Activo}
\author{Manuel Sarmiento Navarro}
\maketitle

\section{Introducción}
Este es un programa que da acceso mediante una interfaz gráfica a las funciones de administración de un Directorio Activo en un servidor remoto. \\
El programa está escrito en \verb|Python 3| y usa las siguientes librerías:
\begin{enumerate}
    \item \verb|logging| para los logs.
    \item \verb|pyad| para la comunicación con el Directorio Activo.
    \item \verb|tkinter| para la interfaz gráfica.
\end{enumerate}

La estructura del directorio en el cuál se trabaja el desarrollo del programa es la siguiente:
\begin{figure}[H]
\centering
\framebox[\textwidth]{%
\begin{minipage}{0.9\textwidth}
\dirtree{%
.1 isat.ad.
.2 doc\DTcomment{Directorio en el que se escribe el presente documento}.
.3 tex.
.3 R.
.3 img.
.2 log\DTcomment{Directorio donde se encuentran los logs}.
.2 src\DTcomment{Directorio donde se encuentra el código fuente}.
.3 GUI.
.3 nucleo.
}
\end{minipage}
}
\caption{Estructura del directorio de trabajo}
\end{figure}

\section{Requisitos}
Para que el programa se ejecute correctamente, será necesario que el equipo en el cual se instale cuente con lo siguiente:
\begin{enumerate}
    \item Windows XP o posterior
    \item \verb|Python 3.x|
\end{enumerate}

\section{Instalación}
Una vez que se cumplan los requisitos el programa se instala ejecutando el \verb|ASDFG.exe|

\section{Instrucciones de uso}
ASDFGFDSgdsgfsd

\section{Descripción del código}
\subsection{nucleo}
\lstinputlisting[language=Python, firstline=10, lastline=28]{../src/nucleo/funcionesAD.py}

\subsection{GUI}

\section{Referencias}
\begin{enumerate}
    \item Acceso remoto a un \emph{AD DS} en una máquina virtual (Virtualbox): \\
        \url{https://www.youtube.com/watch?v=puIaRzZEwyY&t=189s}
    \item Librería de \verb|Python| para administración del Directorio Actiuvo \verb|pyad|: \\
        \url{https://zakird.github.io/pyad/}
    \item Librería de \verb|Python| para hacer interfaces gráficas \verb|tkinter|: \\
        \url{https://docs.python.org/2/library/tkinter.html} \\
        \url{https://www.youtube.com/watch?v=YXPyB4XeYLA}
    \item Librería de \verb|Python| para hacer logs \verb|logging|: \\
        \url{https://docs.python.org/2/library/logging.html}
    \item Conversión de una aplicación de \verb|Python .py| a un ejecutable \verb|.exe|
        \url{https://www.youtube.com/watch?v=UZX5kH72Yx4}
\end{enumerate}

\end{document}
